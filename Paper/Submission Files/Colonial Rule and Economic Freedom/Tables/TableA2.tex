\begin{table}[h!]
\centering
\begin{threeparttable}
\caption{Classification of Countries with Multiple Colonizers}\label{tab:classification}
\begin{tabular*}{\textwidth}{@{\extracolsep{\fill}}lcccc@{\extracolsep{\fill}}}
\hline
\hline
Source & \cite{COLDAT} & \cite{laporta1999quality} & Base Sample \\
Country & (Longest)  &   (Main) &  (Adopted) \\
\hline
Belize     &    Spain          & Britain    &  Britain \\
 Burundi   &    Belgium    &     Germany     &  Germany  \\
 Cameroon  &     Britain    &    Germany    & German  \\
 Canada   &     France      &    Britain   & Britain \\
Ghana   &   Portugal       &   Britain  & Britain \\
Guyana  & Netherlands    &      Britain  & Britain \\
Jamaica   &    Britain     &     Britain  & Britain \\
Libya      &   Italy      &     Italy  & Italy \\
Malaysia    &  Portugal   &       Britain & Britain\\
Mauritania  &       Spain   &   France    & France \\  
Mauritius  &     Britain   &       Britain &  Britain \\
Morocco   &      Spain     &     France    & France \\     
Namibia   &    Germany     &    Britain  & Britain \\
Rwanda    &   Belgium     &     Germany &  Germany  \\
Senegal   &     France    &      France  &  France \\          
Seychelles  &     Britain   &       n/a & Britain \\
Somalia   &    Britain      &    Britain & Britain  \\
South Africa  & Netherlands  &    Britain   & Britain \\
Sri Lanka   &    Britain     &     Britain &  Britain \\
Tanzania   &    Britain      &    German   & German  \\
Togo     &   France    &      France       & France \\
Trinidad \& Tobago     &    Spain    & Britain   & Britain\\
Uruguay     &    Spain     &     Spain   & Spain \\  
\bottomrule
\end{tabular*}
\begin{tablenotes}
\small
\item \textit{Note}:  Excludes 21 countries without EFW data. All the main regressions adopt the Base Sample, following \cite{laporta1999quality}. Main results are not affected if I use the longest colonizer instead, as shown in Table \ref{tab:TabB4}.
\end{tablenotes}
\end{threeparttable}
\end{table}