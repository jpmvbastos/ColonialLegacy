\begin{threeparttable}[h!]
\begin{center}
\small
\caption{Multiple Colonizers, Length of Rule, and Standard Deviation across Areas}
\label{tab:TabB6}
\begin{tabular*}{\textwidth}{@{\extracolsep{\fill}}lccccccc@{\extracolsep{\fill}}}
\hline\hline
                        \textit{Dependent Variable:}
                        &\multicolumn{1}{c}{No Controls}
                        &\multicolumn{1}{c}{Controlling for Colonizer}
                        &\multicolumn{1}{c}{Controlling for Length}\\
            $Std.$ $EFW$&\multicolumn{1}{c}{(1)}&\multicolumn{1}{c}{(2)}&\multicolumn{1}{c}{(3)}\\
\hline
Multiple    &      -0.141         &      -0.169\sym{*}  &     -0.081         \\
            &    (0.087)         &    (0.099)         &     (0.166)         \\
[0.5em]
Centuries of Rule   &                     &                     &     -0.055         \\
            &                     &                     &    (0.059)         \\
[1em]
Colonizer FE & No & Yes & Yes \\
\hline
\(N\)       &         107         &         107         &         107         \\
\(R^{2}\)   &       0.020         &       0.072         &       0.083         \\
\hline\hline
\end{tabular*}
\begin{tablenotes}
\small
\item \textit{Notes}: Robust standard errors in parenthesis. *, **, and *** indicate statistical significance at the 10, 5, and 1\% levels, respectively. Countries with multiple colonizers are classified as colonies of their \textit{main} colonizer, following \cite{laporta1999quality}. Dependent variable is the 2000-2019 average of the within-year standard deviation among the five areas, following \cite{bolen2020does}.
\end{tablenotes}
\end{center}
\end{threeparttable}